\section{Discussione}

Questo studio ha esplorato una serie di video di TikTok condivisi sul social network Twitter nel periodo di tempo di sette giorni 
verso l'inizio del mese di marzo 2022. 
Ogni video è stato classificato riguardo il formato, il contenuto e le emozioni che trasmette, inoltre 
sono stati raccolti e correlati alcuni attributi indicativi dello user engagement tra i quali: numero di views, 
numero di like, di commenti e di condivisioni.

\subsection{Formato video e user engagement}

La maggior parte dei video di TikTok fa largo utilizzo degli hashtag, dalle nostre analisi un numero maggiore di hashtag è 
correlato ad un crescente numero di like.
Recenti ricerche sui Social Media hanno mostrato come l'utilizzo degli hashtag permetta agli utenti di trovare più facilmente i 
contenuti\cite{chang2010new}; una ricerca analoga svolta su Twitter ha mostrato come un utilizzo maggiore degli hashtag porti 
ad un numero più alto di retweet\cite{saxton2015advocatingforchange}.
L'analisi dei differenti hashtag inseriti nelle caption nei video presi in esame mostra che i più popolari hanno una relazione stretta 
con la piattaforma, in particolare con la pagina "per te" di TikTok, alcuni esempi sono \#fyp, \#viral e \#perte come a voler aumentare 
le chance del video di essere inserito nella sezione "per te" degli utenti. 
I restanti hashtag molto popolari riguardano invece il tema della guerra in Ucraina, un tema caldo sopratutto in quei giorni presi in 
esame poiché l'invasione russa era iniziata da poche settimane.

Come visto nella Sezione \ref{risultati} la presenza di testo all'interno dei video è molto alta come lo è quella delle caption, infatti 
quasi ogni video ne ha una, molto popolare è anche l'utilizzo di lingue parlate e di musica di sottofondo, mentre i sottotitoli non 
sono così tanto presenti. Le statistiche al contrario hanno mostrato che quei pochi video che contengono sottotitoli hanno un numero 
maggiore di like, commenti e condivisioni, la presenza di una lingua parlata e di una musica di sottofondo si rivelano altrettanto 
importanti per i valori statistici, contrapposta alla presenza di testo che non influisce minimamente.

La tipologia di video più ricorrente è quella del parlato, seguita dalle compilation, le slideshow e le infografiche; i balletti di TikTok e 
l'acting per quanto due dei formati più gettonati nella piattaforma non sono presenti in numero tanto grande e sono stati accorpati in 
un'unica categoria. Probabilmente non si prestano molto bene ad essere condivisi all'esterno della piattaforma dove non è presente lo stesso 
contesto sociale.

La correlazione dei dati statistici con il tempo passato dalla pubblicazione dei video mostra una correlazione positiva rispetto al numero 
di visualizzazioni e di like, questo fenomeno è del tutto atteso poiché video pubblicati prima hanno avuto più tempo per raggiungere 
un numero maggiore di persone; il tempo non è l'unico fattore che incide sulla fama di un video per questo motivo si possono avere 
video più recenti in grado di performare meglio, questo si può notare nelle correlazioni nulle e negative del numero di commenti e 
condivisioni.

\subsection{Contenuti video ed emozioni}

Il contenuto video più popolare è decisamente quello della guerra in Ucraina, come detto precedentemente è un tema scottante sopratutto nel 
periodo in cui sono stati raccolti i dati ed è normale avere una grande quantità di video che ne parlano; in maniera del tutto 
inaspettata troviamo al secondo posto in gradimento i video a tema gossip e personaggi famosi, la maggior parte di questi verte con gruppi 
musicali sudcoreani, come ad esempio i BTS, oppure sono spezzoni di litigi e momenti imbarazzanti vissuti 
dai partecipanti del Grande Fratello VIP, in onda in quel momento. Al terzo posto troviamo video che parlano di politica interna ed 
estera seguiti a breve distanza da video che parlano di Covid-19 mostrando principalmente un punto di vista No-Vax con 
lamentele all'obbligo vaccinale e all'obbligo di portare le mascherine in vigore in Italia.

L'emozione più popolare che il video suscita nell'osservatore è l'indignazione, questo deriva dal fatto che la maggior parte dei video 
trattano temi relativi alla guerra non con lo scopo di fare informazione, ma al contrario, mostrando un punto di vista principalmente 
filo-Putiniano alimentando il sospetto e la disinformazione.
La seconda emozione più riscontrata è la felicità/risate, alcuni video mostrano direttamente spezzoni comici oppure appositamente 
realizzati per suscitare l'ilarità, 
altri creano felicità motivando l'ascoltatore, altri ancora suscitano uno stato d'animo di wholesomeness tramite video contenenti 
animali carini, eventi passati particolarmente piacevoli oppure eventi caritatevoli.

\subsection{Limitazioni e ricerche future}

Nell'approccio utilizzato in questa indagine è possibile identificare diverse limitazioni; per prima cosa la provenienza dei dati, 
i video analizzati sono ricavati da tweet condivisi su Twitter in un periodo di tempo di sette giorni, questo periodo può essere poco 
esaustivo per estrarre al meglio le tipologie, i contenuti e le tematiche che generalmente hanno successo in un contesto multi piattaforma. 
Il periodo in cui si è svolta la ricerca inoltre è in grado d'introdurre una sorta di bias spostando l'attenzione solo su contenuti a 
tema guerra e, di conseguenza, emozioni negative; nelle ricerche future si dovrebbero raccogliere i dati in un periodo di tempo maggiore e 
prestare attenzione agli eventi di scala globale che avvengono in quel momento. 
La seconda limitazione rilevata riguarda il processo di classificazione dei video che è stato eseguito da un solo coder in diverse sessioni 
nell'arco di alcuni giorni; in futuro sarebbe opportuno ampliare questo processo facendo analizzare gli stessi video a due o più coder 
in modo da discutere e risolvere tutte le discrepanze che possono emergere ed ottenere una classificazione più generale.

Il modo in cui si è calcolata la correlazione dei parametri di user engagement con il numero di giorni trascorsi dalla creazione del 
contenuto risulta essere un'ulteriore restrizione; analisi future potrebbero concentrarsi sulla variazione delle metriche nel tempo eseguendo 
il processo di raccolta ogni giorno in modo da poter vedere come il passare del tempo influisce su di esse.

Un ultimo spunto di ricerca consiste nel capire se i video condivisi su Twitter hanno l'effetto di aumentare lo user engagement nella 
piattaforma originale, ossia TikTok; per questa ricerca si consiglia di impostare uno studio non solo dei contenuti condivisi, ma anche 
di video della stessa tipologia che non sono stati mai condivisi su Twitter in modo da poterli paragonare.