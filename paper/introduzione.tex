\section{Introduzione}

I Social Network al giorno d'oggi assumono un ruolo molto importante nella vita di una parte della popolazione sempre più connessa 
online permettendo nuovi approcci di comunicazione a distanza e consentendo un approccio del tutto nuovo alla fruizione dei 
contenuti e delle informazioni che un tempo erano relegate a mezzi di comunicazione non digitali. Twitter 
è emerso negli anni come un popolare sistema sociale in cui gli utenti hanno la possibilità di discutere qualsiasi tema sia di 
loro gradimento, da semplici battute alle news fino a eventi che riguardano strettamente la loro vita personale\cite{benevenuto2010detecting}. 
Con oltre 217 milioni di utenti attivi giornalmente e oltre 500 milioni di tweet condivisi al giorno Twitter ha rivoluzionato 
il concetto del micro-blogging ed è diventato rapidamente uno dei principali sistemi per la fruizione di notizie in tempo reale. 

Dall'altro lato TikTok è considerato un Social Network ancora emergente; lanciato in Cina nel 2017, ha rapidamente ottenuto 
popolarità in tutto il mondo con olre 800 milioni di utenti attivi al mese e oltre 2 miliardi di download delle sue applicazioni 
mobile\cite{li2021communicating}.
La sua popolarità è dovuta al fatto che la piattaforma si basa interamente sui contenuti video di brevissima durata, mediamente 
dai 15 ai 60 secondi, di facile comprensione accessibili immediatamente dalla maggior parte della community. 
La vera forza di TikTok risiede nel suo algoritmo di raccomandazione video, basato sul concetto di scrolling infinito, che da all'utente 
la sensazione di avere sempre qualcosa di nuovo da vedere tenendolo incollato alla piattaforma anche per diverse ore.

Un fenomeno recentemente emerso dalla grande popolarità di questi sistemi è quello del platform hacking; per hacking della piattaforma 
si intende l'insieme di tutte quelle pratiche, coscienti o meno, messe in atto dai suoi utenti per sfruttare a loro vantaggio i complessi 
algoritmi e ottenere un qualche reward, nella maggior parte dei casi si tratta di un incremento della visibilità dei contenuti prodotti 
dall'utente. 
Fra le strategie di platform hacking troviamo la condivisione di contenuti su piattaforme esterne a quella in cui sono stati pubblicati 
al fine di aumentare le statistiche del contenuto coinvolgendo un pubblico completamente diverso.

In questo articolo tratteremo il tema della condivisione multi-piattaforma analizzando contenuti video prodotti su TikTok e condivisi 
su Twitter.
Inizialmente affronteremo il problema della raccolta di parametri di user engagement da contenuti multimediali sviluppando 
uno strumento in grado di agevolare parte del processo; passeremo poi alla categorizzazione manuale di ogni video in una serie di 
classi evolute anch'esse durante il processo. Infine analizzeremo tutti i dati raccolti per cercare possibili correlazioni al loro 
interno.