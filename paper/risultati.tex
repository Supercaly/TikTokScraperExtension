\section{Risultati}
\label{risultati}

L'analisi dei dati si è svolta a partire dal 22 maggio 2022 e ha elaborato un totale di 301 tweet condivisi nel periodo 
che va dal 3 al 17 Marzo 2022; purtroppo al momento dello studio non tutti i video precedentemente raccolti 
ero disponibili, fra questi 49 risultano rimossi dalla piattaforma oppure, oscurati dal loro autore originale lasciandoci
con 252 video con cui lavorare. 

Poiché i video di TikTok a cui abbiamo accesso provengono da condivisioni su Twitter alcuni di questi risultano essere 
condivisi molteplici volte, selezioniamo quindi 181 video unici. 
In totale tutti i video hanno 190414036 views (Media = 764715, SD=2243165), 19238445 likes (Media = 76343, SD = 349829), 
262750 commenti (Media = 1043, SD = 3059) e 1073749 condivisioni (Media = 4261, SD = 9974). 
La loro lunghezza varia da 1 secondo a 10 minuti per un totale di 280 minuti di video (Media = 67s, SD = 91s), di questi 
solo 80 (31,7\%) hanno una durata superiore ad 1 minuto. 
Fra questi video il 14,68\% (37) utilizzano sottotitoli che trascrivono o traducono i dialoghi presenti all'interno, il 
63,10\% (159) contengono del testo aggiuntivo, il 53,17\% (135) ha delle lingue parlate al loro interno, il 55,16\% (139) 
ha una musica di sottofondo e il 76,19\% (192) presenta una caption inserita come descrizione in aggiunta al video.
La maggior parte delle caption contiene anche degli hashtag (908 in totale, range=0-29) che sono stati registrati e catalogati
separatamente.

Come detto nella Sezione \ref{sec:metodi} ogni video è stato visionato più volte al fine di assegnare una singola classe per 
formato video, contenuto del video ed emozioni scaturite dalla visione.
Per quanto riguarda il formato, i video sono stati assegnati alle nove categorie rilevate nel seguente modo: il 
formato più popolare è il Parlato con il 23,02\%(58) seguito da Compilation con il 15,48\%(39), le Slideshow con il 14,29\%(36), 
le Infografiche con l'11,90\%(30), la Presa diretta con l'11,51\%(29), i programmi TV/film con il 9,92\%(25), l'Acting 
con il 10,71\%(27), le News con l'1,98\%(5) e le Reaction con l'1,19\%(3).

Di seguito vediamo i risultati dell'inserimento nelle quindici categorie di contenuti video rilavate: al primo posto troviamo 
il contenuto che tratta la guerra in Ucraina con il 27,78\%(70) dei video, il contenuto vip/gossip con il 15,08\%(38), 
il tema politico con l'11,51\%(29), i video in cui sono presenti performance musicali con l'8,33\%(21), i video di umorismo con 
il 7,94\%(20), quelli che trattano il tema del virus Covid-19 con il 7,54\%(19), i video dal contenuti sconosciuto e non analizzabile 
con il 6,75\%(17), i video motivazionali con il 5,56\%(14), gli omaggi a personaggi famosi o a eventi passati con il 2,38\%(6), 
i video dal contenuti personale con il 2,38\%(6), le promozioni di contenuti personali con l'1,98\%(5), i video a tema sportivo 
con l'1,19\%(3), i video contenenti animali domestici o selvatici con lo 0,79\%(2), i contenuti di stampo complottista con lo 
0,40\%(1) e i gameplay con lo 0,40\%(1).

Infine esaminiamo le emozioni provate dal coder durante la visione di ogni video e catalogate in undici categorie: al primo posto 
troviamo l'indignazione con il 27,78\% (70), la felicità/risate con il 23,81\% (60), l'ammirazione con il 17,86\% (45), i contenuti 
sconosciuti e non analizzabili con il 8,73\% (22), la paura/rabbia/tristezza con il 7,54\% (19), l'hype con il 3,97\% (10), 
il sospetto con il 3,97\% (10), lo stupore con il 2,38\% (6), la nostalgia con l'1,59\% (4), l'informazione con l'1,59\% (4) e 
la speranza con il 0,79\% (2).

Una volta analizzati i risultati del processo di categorizzazione dei diversi video ci concentriamo sull'analisi degli hashtag riportati
all'interno delle caption che sono stati raccolti e catalogati separatamente. In totale sono stati raccolti 907 hashtag, di cui 630 
sono unici; i dieci hashtag più popolari sono: \#perte (19), \#fyp (17), \#viral (16), \#putin (15), \#russia (15), \#foryou (13), 
\#ucraina (13), \#guerra (12), \#nato (10), \#neiperte (10). 
Come possiamo ben vedere gli hashtag più popolari riguardano la pagina "per te" di TikTok, questo è dovuto alla tendenza degli utenti 
di voler massimizzare le statistiche dei loro video "sfruttando" l'algoritmo della piattaforma a loro favore, fra i diversi hack 
contro l'algoritmo uno molto popolare è proprio l'utilizzo strategico degli hashtag.
Questo concetto può essere esteso anche ai restanti hashtag che trattano il tema della guerra in Ucraina, un tema decisamente potente 
ed efficace per sfruttare l'algoritmo ed avere una maggiore visibilità.

Guardando le statistiche raccolte emerge che i video hanno un numero maggiore di like, commenti e condivisioni quando al loro interno 
sono presenti sottotitoli rispetto a quando non sono utilizzati; analogamente i video contenenti una qualsiasi lingua parlata 
hanno un maggior numero di commenti e condivisioni, la stessa cosa si nota nei video in cui è presente una musica di sottofondo i 
quali hanno molti più commenti e like rispetto ai video in cui non è utilizzata alcuna musica. 
Al contrario la presenza di testo all'interno del video non influisce in alcun modo sulle metriche riscontrando un numero 
di like, commenti e condivisioni maggiori quando il testo non è presente; questo fenomeno si scontra con quanto visto 
in precedenza considerando che più della metà di tutti i video contiene al suo intero del testo.
Infine è interessante notare come i video contenenti emozioni allarmanti e negative abbiano ricevuto complessivamente 
un numero maggiore di like e commenti rispetto a video con emozioni positive.

Successivamente prendiamo in esame i valori dello user engagement in correlazione con il tempo passato dalla pubblicazione 
dei video, da 83 fino a 90 giorni. 
Come si può immaginare il numero di views e di like tende ad aumentare maggiore è la distanza temporale di pubblicazione (correlazione 
positiva dello 0.25 e 0.14); al contrario il numero di condivisioni rimane quasi del tutto costante (correlazione dello 0.02) e 
addirittura il numero di commenti tende a diminuire (correlazione negativa dello -0.3).

Infine approfondiamo i video in relazione alle loro condivisioni su Twitter. 
Nel periodo di tempo preso in considerazione i 301 video sono stati condivisi su Twitter da 146 account unici, 
fra questi tre account che hanno condiviso un numero maggiore di video, rispettivamente 20, 16 e 11 video (Media = 2.04, SD = 2.51). 
Di seguito concentriamoci solo sui dieci video più condivisi su Twitter; in questi il formato video più condiviso è quello 
dell'infografica (30\%), il contenuto video è quello della guerra in Ucraina (40\%) mentre l'emozione più condivisa è 
l'indignazione (40\%). 
Il numero di like, commenti, condivisioni e views è positivamente correlata con il numero di condivisioni su Twitter 
(0,38, 0,18, 0,57, 0,50), mentre la durata del video non sembra avere alcuna relazione. 
Fra i dieci video solo uno contiene sottotitoli, mentre quasi tutti contengono del testo, una lingua parlata e la caption, 
la musica è presente in meno della metà dei video. 
