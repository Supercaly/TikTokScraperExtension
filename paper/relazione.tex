\documentclass[fleqn, 10pt]{SelfArx}
\usepackage[utf8]{inputenc}
\usepackage[T1]{fontenc}
\usepackage[italian]{babel}
\usepackage{lmodern}
\usepackage{csquotes}
\usepackage{subcaption}
\usepackage{graphicx}
\usepackage{url}

\setlength{\columnsep}{0.55cm} % Distance between the two columns of text
\setlength{\fboxrule}{0.75pt} % Width of the border around the abstract

\definecolor{color1}{RGB}{0,0,90} % Color of the article title and sections
\definecolor{color2}{RGB}{0,20,20} % Color of the boxes behind the abstract and headings

\usepackage{hyperref} % Required for hyperlinks

\hypersetup{
	hidelinks,
	colorlinks,
	breaklinks=true,
	urlcolor=color2,
	citecolor=color1,
	linkcolor=color1,
	bookmarksopen=false,
	pdftitle={Analisi quantitativa e qualitativa di video TikTok condivisi su Twitter},
	pdfauthor={Lorenzo Calisti},
}


\JournalInfo{Etica della comunicazione digitale - Laurea Magistrale in Informatica Applicata - Università degli Studi di Urbino Carlo Bo} % Journal information
% \Archive{Data di pubblicazione 10/02/2022 - DOI: \href{https://doi.org/10.5281/zenodo.6036183}{10.5281/zenodo.6036183}} % Informazioni che verranno inserite dal Docente in fase di pubblicazione

\PaperTitle{Analisi quantitativa e qualitativa di video TikTok condivisi su Twitter} % Article title

\Authors{Lorenzo Calisti\textsuperscript{1}*}
% al fine di poter gestire la procedura di submission sul repository pubblico (nell'affiliazione viene chiarito il ruolo)
\affiliation{\textsuperscript{1}\textit{Laurea Magistrale in Informatica Applicata, Università degli Studi di Urbino Carlo Bo, Urbino, Italia}} % Author affiliation
% \affiliation{\textsuperscript{2}\textit{Docente di Programmazione per l'Internet of Things, Università degli Studi di Urbino Carlo Bo, Urbino, Italia}} % Author affiliation
\affiliation{*\textbf{Corresponding author}: l.calisti@campus.uniurb.it} % Corresponding author

\Keywords{Twitter -- TikTok -- Analisi dei contenuti } % Keywords - if you don't want any simply remove all the text between the curly brackets
\newcommand{\keywordname}{Keywords} % Defines the keywords heading name

\Abstract{
Al giorno d'oggi i Social Network rappresentano una parte importante nella vita di buona parte della popolazione mondiale sempre più 
connessa, fra questi vi sono sicuramente i social Tiwitter e TikTok; il primo ha rivoluzionato il mondo del micro-blogging divenendo il 
mezzo d'informazione primario per molte persone, il secondo ha trasformato l'approccio al contenuto multimediale grazie 
a video di brevissima durata e ad un algoritmo in grado di consigliare il video migliore nel momento giusto.
Questo articolo propone uno studio dei contenuti, dei formati e delle emozioni rilevati in una serie di video originariamente 
pubblicati su TikTok ed in seguito ricondivisi su Twitter. Per l'occasione è anche stato sviluppato un tool in grado di semi-automatizzare 
la fase di raccolta delle metriche di user engagement.
}

\begin{document}

\maketitle

\section{Introduzione}

I Social Network al giorno d'oggi assumono un ruolo molto importante nella vita di una parte della popolazione sempre più connessa 
online permettendo nuovi approcci di comunicazione a distanza e consentendo un approccio del tutto nuovo alla fruizione dei 
contenuti e delle informazioni che un tempo erano relegate a mezzi di comunicazione non digitali. Twitter 
è emerso negli anni come un popolare sistema sociale in cui gli utenti hanno la possibilità di discutere qualsiasi tema sia di 
loro gradimento, da semplici battute alle news fino a eventi che riguardano strettamente la loro vita personale\cite{benevenuto2010detecting}. 
Con oltre 217 milioni di utenti attivi giornalmente e oltre 500 milioni di tweet condivisi al giorno Twitter ha rivoluzionato 
il concetto del micro-blogging ed è diventato rapidamente uno dei principali sistemi per la fruizione di notizie in tempo reale. 

Dall'altro lato TikTok è considerato un Social Network ancora emergente; lanciato in Cina nel 2017, ha rapidamente ottenuto 
popolarità in tutto il mondo con olre 800 milioni di utenti attivi al mese e oltre 2 miliardi di download delle sue applicazioni 
mobile\cite{li2021communicating}.
La sua popolarità è dovuta al fatto che la piattaforma si basa interamente sui contenuti video di brevissima durata, mediamente 
dai 15 ai 60 secondi, di facile comprensione accessibili immediatamente dalla maggior parte della community. 
La vera forza di TikTok risiede nel suo algoritmo di raccomandazione video, basato sul concetto di scrolling infinito, che da all'utente 
la sensazione di avere sempre qualcosa di nuovo da vedere tenendolo incollato alla piattaforma anche per diverse ore.

Un fenomeno recentemente emerso dalla grande popolarità di questi sistemi è quello del platform hacking; per hacking della piattaforma 
si intende l'insieme di tutte quelle pratiche, coscienti o meno, messe in atto dai suoi utenti per sfruttare a loro vantaggio i complessi 
algoritmi e ottenere un qualche reward, nella maggior parte dei casi si tratta di un incremento della visibilità dei contenuti prodotti 
dall'utente. 
Fra le strategie di platform hacking troviamo la condivisione di contenuti su piattaforme esterne a quella in cui sono stati pubblicati 
al fine di aumentare le statistiche del contenuto coinvolgendo un pubblico completamente diverso.

In questo articolo tratteremo il tema della condivisione multi-piattaforma analizzando contenuti video prodotti su TikTok e condivisi 
su Twitter.
Inizialmente affronteremo il problema della raccolta di parametri di user engagement da contenuti multimediali sviluppando 
uno strumento in grado di agevolare parte del processo; passeremo poi alla categorizzazione manuale di ogni video in una serie di 
classi evolute anch'esse durante il processo. Infine analizzeremo tutti i dati raccolti per cercare possibili correlazioni al loro 
interno.
\section{Metodologia e Materiali}
\label{sec:metodi}

\subsection{Raccolta dei dati}

La seguente ricerca sviluppa uno studio di tipo quantitativo e qualitativo su una serie di video pubblicati sul Social Network TikTok
e condivisi da diversi utenti su Twitter in un periodo di tempo che val dal 3 marzo 2022 al 17 marzo 2022. 
I tweet condivisi sono stati esportati utilizzando le API ufficiali messe a disposizione da Twitter\cite{twitterapi} che 
consentono a tutti gli utenti registrati di scaricare una lista di tweet prodotti in un periodo di tempo specifico. 
Le API ufficiali offrono un gran numero di informazioni riguardanti i tweet ricercati, fra queste abbiamo: il testo contenuto 
nel tweet, la sua data di creazione, gli hashtag presenti, e diverse altre informazioni riguardanti l'utente che lo ha creato.

Quando si utilizzano le API di Twitter i dati sono ritornati nel formato JSON (JavaScript Object Notation), il primo passaggio è 
quindi quello di convertirli in un formato più facile da elaborare; tutte le analisi in questa ricerca sono state svolte tramite 
il software Google Sheet\cite{googlesheet}, uno strumento gratuito che permette di lavorare con i dati formattati in tabelle. I dati raw
scaricati sono stati filtrati su Google Sheet in modo da estrarre solo le colonne di maggiore interesse e semplificare i 
passaggi successivi.

Una volta ottenuti tutti i dati di partenza ha avuto inizio il processo di raccolta delle statistiche di user engagement presenti sul 
video originale di TikTok.
La raccolta di questi dati è stata svolta im maniera semi automatica tramite l'utilizzo di un tool di web scraping realizzato 
appositamente per questa ricerca. 
Lo strumento dal nome TikTokScraperExtension\cite{scraper} consiste in una semplice estensione del browser Google Chrome 
programmata nel linguaggio di programmazione JavaScript che fa largo utilizzo delle API DOM (Document Object Model) messe a disposizione dal browser 
per eseguire web scraping di tutti i contenuti che sono normalmente visibili aprendo il video nella pagina web di TikTok come ad esempio: 
la lunghezza del video, la caption e il numero di like, commenti e condivisioni. 
Il processo è semi automatico poiché l'operatore deve comunque aprire ogni video in una scheda separata di Google Chrome, 
attendere l'auto-play e premere un singolo pulsante che copierà tutte le informazioni ricercate direttamente all'interno della clipboard; 
in seguito sarà necessario incollare questi dati in una colonna del foglio di lavoro di Google Sheet.
I dati raccolti in questo modo sono però formattati sotto forma di stringhe JSON e quindi devono essere processati tramite delle 
espressioni regolari per estrarre i veri valori prima del loro utilizzo; questo passaggio è stato eseguito tramite delle funzioni 
native di Google Sheet.

Il processo automatico è in grado di raccogliere solo i dati visibili negli elementi della pagina web di TikTok, alcune informazioni 
però sono codificate direttamente all'interno dei video che devono essere analizzati manualmente da un operatore, queste variabili sono: 
la presenza di sottotitoli, di testo, di linguaggi parlati e di musiche di sottofondo. Ognuna di queste variabili sono di tipo 
booleano e vengono codificate come presente (TRUE) o assente (FALSE). Nella tabella \ref{tab:video_var} è possibile vedere tutte 
le variabili raccolte manualmente dall'operatore oppure mediante il processo semi automatizzato.

\begin{table*}[htb]
    \centering
    \small
    \begin{tabular}{l c c r}
        \hline
        Variabile & Descrizione & N (\%) \\
        \hline
        Views & Il numero di visualizzazioni del video & - \\
        Likes & Il numero di like del video & - \\
        Commenti & Il numero di commenti del video & - \\
        Condivisioni & Il numero di condivisioni del video & - \\
        Lunghezza & La lunghezza del video & - \\
        Hashtag & Gli hashtag inseriti nella caption se presenti & - \\
        Sottotitoli & Il video include del testo che traduce & 37 (14,68\%) \\
                    & o trascrive il dialogo & \\
        Testo & Il video include del testo che aumenta & 159 (63,10\%) \\
              & in qualche modo l'efficacia del messaggio & \\
        Lingua parlata & Nel video è parlata una lingua & 134 (53,17\%) \\
        Musica & Il video include una musica di sottofondo & 139 (55,16\%) \\
        Caption & Un messaggio è inserito come descrizione in aggiunta al video & 192 (76,19\%) \\  
        \hline
    \end{tabular}    
    \caption{Variabili che indicizzano il video di TikTok. Le variabili Sottotitoli, Testo, Lingua parlata, Musica e Caption
    sono state codificate come presente (1) o assente (0). }
    \label{tab:video_var}
\end{table*}

\subsection{Analisi dei video di TikTok}

Una volta raccolte tutte le informazioni analitiche nell'interfaccia di TikTok oppure direttamente all'interno dei video ci siamo 
concentrati sull'analisi dei formati, dei contenuti e delle emozioni.
Ogni video è stato visionato numerose volte da un singolo coder in diverse sessioni nell'arco di alcuni giorni cercando di inserirlo 
ogni volta nella categoria più opportuna, questo processo è stato particolarmente dispendioso in termini di tempo poiché man mano che 
i video venivano categorizzati anche le classi venivano aggiornate rendendo necessario ricontrollare tutti i video passati.

Il processo di classificazione è stato eseguito inizialmente concentrandosi solamente nell'identificare il formato video, 
le categorie delineate sono mutualmente esclusive e un video può appartenere solo ad una di queste. Il processo ha trovato 
le seguenti categorie: acting, infografica, news, parlato, slideshow, presa diretta, reaction, compilation, TV/Film. 
La tabella \ref{tab:video_type} mostra nel dettaglio le tipologie di formato video rilevate.

\begin{table*}[htb]
    \centering
    \small
    \begin{tabular}{l c c r}
        \hline
        Formato & Descrizione & Esempi & N (\%) \\
        \hline
        Acting & Un video in cui sono presenti balletti  & https://vm.tiktok.com/ZMLUM1Hky &  27 (10,71\%) \\
               & di TikTok oppure persone che recitano & https://vm.tiktok.com/ZMLUB3QNE & \\
               & appositamente per il video &  & \\
        Infografica & Un video in cui le informazioni sono & https://vm.tiktok.com/ZMLUkV2JU & 30 (11,90\%) \\
                    & presentate tramite immagini, disegni, & https://vm.tiktok.com/ZMLUSA9Qn & \\
                    & animazioni, illustrazioni o grafici & & \\
        News &	Un video che presenta informazioni su & https://vm.tiktok.com/ZMLU9EV3Y & 5 (1,98\%) \\
             &  fatti accaduti recentemente & https://vm.tiktok.com/ZMLy7pKU4 & \\
        Parlato & Un video in cui ci sono persone & https://vm.tiktok.com/ZMLUV1jSp & 58 (23,02\%) \\
                & inquadrate mentre parlano & https://vm.tiktok.com/ZMLU47Fjt & \\
        Slideshow &	Un video composto da una carrellata di & https://vm.tiktok.com/ZMLUSY2MH & 36 (14,29\%) \\
                  & fotografie o immagini statiche & https://vm.tiktok.com/ZMLUBELm9 & \\
        Presa diretta & Un video che mostra degli eventi in diretta& https://vm.tiktok.com/ZMLUF31S3 & 29 (11,51\%) \\
                      & & https://vm.tiktok.com/ZMLUYDvrh & \\
        Reaction & Un video in cui persone reagiscono ad & https://vm.tiktok.com/ZMLUYVdvk & 3 (1,19\%) \\
                 & altri contenuti multimediali & https://vm.tiktok.com/ZMLUqneqx & \\
        Compilation & Un video composto da una carrellata & https://vm.tiktok.com/ZMLU2x7rV & 39 (15,48\%) \\
                    & di video montati assieme & https://vm.tiktok.com/ZMLUj87Lt & \\
        TV/Film & Un video contenente spezzoni di film o & https://vm.tiktok.com/ZMLU2PBBs & 25 (9,92\%) \\
                & programmi andati in onda in TV & https://vm.tiktok.com/ZMLU2TSGv & \\
        \hline
    \end{tabular}
    \caption{Formato video di TikTok. Le categorie sono mutualmente esclusive, ogni video appartiene a una sola.}
    \label{tab:video_type}
\end{table*}

Per quanto riguarda i contenuti video la categorizzazione è avvenuta in maniera del tutto analoga rilevando le seguenti classi: 
Sconosciuto, Vip/Gossip, Guerra, Promo, Motivazionale, Umorismo, Omaggio, Complottismo, Animali, Musica, Covid-19, 
Politica, Personale, Gameplay, Sport. 
La tabella \ref{tab:video_content} mostra nel dettaglio ogni categoria video individuata.

\begin{table*}[htb]
    \centering
    \small
    \begin{tabular}{l c c r}
        \hline
        Contenuto & Descrizione & Esempi & N (\%) \\
        \hline
        Sconosciuto & Un video con contenuti non analizzabili & https://vm.tiktok.com/ZMLDV9cKB & 17 (6,75\%) \\
                    & & https://vm.tiktok.com/ZMLDVneCS & \\
        Vip/Gossip & Un video contenente personaggi famosi & https://vm.tiktok.com/ZMLDWSasV& 38 (15,08\%) \\
                   & o fatti di gossip & https://vm.tiktok.com/ZMLUBD6k1 & \\ 
        Guerra & Un video con contenuti riguardanti & https://vm.tiktok.com/ZMLDKTUQ9 & 70 (27,78\%) \\
               & la guerra in Ucraina e l'invasione Russa & https://vm.tiktok.com/ZMLUFEKhy& \\
        Promo & Un video con lo scopo di promuovere & https://vm.tiktok.com/ZMLDVqcjx & 5 (1,98\%) \\
              & contenuti personali & https://vm.tiktok.com/ZMLUVY6VB & \\
        Motivazionale & Un video con lo scopo di motivare l'utente & https://vm.tiktok.com/ZMLD7XRH8 & 14 (5,56\%) \\
                     & & https://vm.tiktok.com/ZMLUVYgVc & \\
        Umorismo & Un video di tipo umoristico & https://vm.tiktok.com/ZMLD3hGCq & 20 (7,94\%) \\
                 & & https://vm.tiktok.com/ZMLUYVdvk & \\
        Omaggio & Un video con lo scopo di omaggiare & https://vm.tiktok.com/ZMLU2x7rV & 6 (2,38\%) \\
                & un personaggio famoso o un evento passato & https://vm.tiktok.com/ZMLU2TSGv & \\
        Complottismo & Un video con contenuti complottisti & https://vm.tiktok.com/ZMLU2PBBs & 1 (0,40\%) \\
                & & & \\
        Animali & Un video contenente animali & https://vm.tiktok.com/ZMLUj1DVM & 2 (0,79\%) \\
                & & https://vm.tiktok.com/ZMLfRXYML & \\
        Musica & Un video contenente performance musicali & https://vm.tiktok.com/ZMLUYDvrh & 21 (8,33\%) \\
                & & https://vm.tiktok.com/ZMLUYSB7X & \\
        Covid-19 & Un video riguardo l'emergenza Covid-19 & https://vm.tiktok.com/ZMLU2bLxk & 19 (7,54\%) \\
                & & https://vm.tiktok.com/ZMLUUAAyB & \\
        Politica & Un video che tratta contenuti politici & https://vm.tiktok.com/ZMLUhmwJn & 29 (11,51\%) \\
                & & https://vm.tiktok.com/ZMLUkV2JU & \\
        Personale & Un video con contenuti personali & https://vm.tiktok.com/ZMLUB3QNE & 6 (2,38\%) \\
                & & https://vm.tiktok.com/ZMLUpePgW & \\
        Gameplay & Un video in cui si mostra un gameplay di un videogioco & https://vm.tiktok.com/ZMLy7kHcq & 1 (0,40\%) \\
                & & & \\
        Sport & Un video contenente sport & https://vm.tiktok.com/ZMLUSAvJc & 3 (1,19\%) \\
                & & https://vt.tiktok.com/ZSd1GJv4U & \\
        \hline
    \end{tabular}
    \caption{Contenuti dei video di TikTok. Le categorie sono mutualmente esclusive, ogni video appartiene a una sola.}
    \label{tab:video_content}
\end{table*}

L'ultimo passaggio consiste nel trovare le classi riguardanti le emozioni che ogni video provoca in chi lo visiona; quest'ultimo 
processo può essere considerato come il più complicato da classificare poiché riguarda profondamente la soggettività dell'individuo. 
Dopo numerosi tentativi sono state individuate le seguenti categorie di emozioni: Sconosciuto, Hype, Sospetto, Stupore, Nostalgia, 
Ammirazione, Indignazione, Felicità/Risate, Speranza, Paura/Rabbia/Tristezza, Informazione.
Nella tabella \ref{tab:video_emotion} è possibile vedere tutte le emozioni registrate.

\begin{table*}[htb]
    \centering
    \small
    \begin{tabular}{l c c r}
        \hline
        Emozioni & Descrizione & Esempi & N (\%) \\
        \hline
        Sconosciuto & Le emozioni nel video non sono analizzabili & https://vm.tiktok.com/ZMLDV9cKB & 22 (8,73\%) \\
        & & https://vm.tiktok.com/ZMLDVneCS& \\
        Hype & Il video trasmette emozioni di hype rispetto & https://vm.tiktok.com/ZMLDVqcjx& 10 (3,97\%) \\
        & gli eventi, i personaggi o i prodotti mostrati & https://vm.tiktok.com/ZMLUVY6VB& \\
        Sospetto & Il video trasmette emozioni di sospetto & https://vm.tiktok.com/ZMLDarydw& 10 (3,97\%) \\
        & riguardo gli eventi mostrati & https://vm.tiktok.com/ZMLDqtXWX & \\
        Stupore & Il video trasmette emozioni di stupore & https://vm.tiktok.com/ZMLDWSasV& 6 (2,38\%) \\
        & & https://vm.tiktok.com/ZMLUBD6k1 & \\
        Nostalgia & Il video trasmette emozioni di nostalgia & https://vm.tiktok.com/ZMLU2TSGv& 4 (1,59\%) \\
        & & https://vm.tiktok.com/ZMLUj87Lt & \\
        Ammirazione & Il video trasmette emozioni di ammirazione & https://vm.tiktok.com/ZMLUj1DVM& 45 (17,86\%) \\
        & verso fatti o persone mostrate & https://vm.tiktok.com/ZMLUY2qA4 & \\
        Indignazione & Il video trasmette emozioni di sdegno & https://vm.tiktok.com/ZMLU2KXuP& 70 (27,78\%) \\
        & riguardo gli eventi mostrati & https://vm.tiktok.com/ZMLUMkYxU& \\
        Felicità/Risate & Il video trasmette emozioni di felicità & https://vm.tiktok.com/ZMLUM1Hky& 60 (23,81\%) \\
        & & https://vm.tiktok.com/ZMLUYSB7X& \\
        Speranza & Il video trasmette emozioni di speranza & https://vm.tiktok.com/ZMLUV5vrD& 2 (0,79\%) \\
        & & https://vm.tiktok.com/ZMLymkyXD& \\
        Paura/Rabbia/Tristezza & Il video trasmette emozioni di paura, & https://vm.tiktok.com/ZMLU2x7rV& 19 (7,54\%) \\
        & rabbia o tristezza riguardo i fatti mostrati & https://vm.tiktok.com/ZMLUURA1a& \\
        Informazione & Dopo aver guardato il video l'utente si & https://vm.tiktok.com/ZMLUAreDA& 4 (1,59\%) \\
        & sente informato riguardo i fatti mostrati & https://vm.tiktok.com/ZMLypBxSm& \\
        \hline
    \end{tabular}    
    \caption{Emozioni trasmesse dai video di TikTok. Le categorie sono mutualmente esclusive, ogni video appartiene a una sola.}
    \label{tab:video_emotion}
\end{table*}

Notiamo infine che sia per i contenuti che per le emozioni è presente la categoria "sconosciuto", questo è dovuto al fatto 
che alcuni video riportano testi e voci interamente in lingue diverse dall'italiano o dall'inglese e per questo motivo non 
è stato possibile categorizzarli correttamente.
\section{Risultati}
\label{risultati}

L'analisi dei dati si è svolta a partire dal 22 maggio 2022 e ha elaborato un totale di 301 tweet condivisi nel periodo 
che va dal 3 al 17 Marzo 2022; purtroppo al momento dello studio non tutti i video precedentemente raccolti 
ero disponibili, fra questi 49 risultano rimossi dalla piattaforma oppure, oscurati dal loro autore originale lasciandoci
con 252 video con cui lavorare. 

Poiché i video di TikTok a cui abbiamo accesso provengono da condivisioni su Twitter alcuni di questi risultano essere 
condivisi molteplici volte, selezioniamo quindi 181 video unici. 
In totale tutti i video hanno 190414036 views (Media = 764715, SD=2243165), 19238445 likes (Media = 76343, SD = 349829), 
262750 commenti (Media = 1043, SD = 3059) e 1073749 condivisioni (Media = 4261, SD = 9974). 
La loro lunghezza varia da 1 secondo a 10 minuti per un totale di 280 minuti di video (Media = 67s, SD = 91s), di questi 
solo 80 (31,7\%) hanno una durata superiore ad 1 minuto. 
Fra questi video il 14,68\% (37) utilizzano sottotitoli che trascrivono o traducono i dialoghi presenti all'interno, il 
63,10\% (159) contengono del testo aggiuntivo, il 53,17\% (135) ha delle lingue parlate al loro interno, il 55,16\% (139) 
ha una musica di sottofondo e il 76,19\% (192) presenta una caption inserita come descrizione in aggiunta al video.
La maggior parte delle caption contiene anche degli hashtag (908 in totale, range=0-29) che sono stati registrati e catalogati
separatamente.

Come detto nella Sezione \ref{sec:metodi} ogni video è stato visionato più volte al fine di assegnare una singola classe per 
formato video, contenuto del video ed emozioni scaturite dalla visione.
Per quanto riguarda il formato, i video sono stati assegnati alle nove categorie rilevate nel seguente modo: il 
formato più popolare è il Parlato con il 23,02\%(58) seguito da Compilation con il 15,48\%(39), le Slideshow con il 14,29\%(36), 
le Infografiche con l'11,90\%(30), la Presa diretta con l'11,51\%(29), i programmi TV/film con il 9,92\%(25), l'Acting 
con il 10,71\%(27), le News con l'1,98\%(5) e le Reaction con l'1,19\%(3).

Di seguito vediamo i risultati dell'inserimento nelle quindici categorie di contenuti video rilavate: al primo posto troviamo 
il contenuto che tratta la guerra in Ucraina con il 27,78\%(70) dei video, il contenuto vip/gossip con il 15,08\%(38), 
il tema politico con l'11,51\%(29), i video in cui sono presenti performance musicali con l'8,33\%(21), i video di umorismo con 
il 7,94\%(20), quelli che trattano il tema del virus Covid-19 con il 7,54\%(19), i video dal contenuti sconosciuto e non analizzabile 
con il 6,75\%(17), i video motivazionali con il 5,56\%(14), gli omaggi a personaggi famosi o a eventi passati con il 2,38\%(6), 
i video dal contenuti personale con il 2,38\%(6), le promozioni di contenuti personali con l'1,98\%(5), i video a tema sportivo 
con l'1,19\%(3), i video contenenti animali domestici o selvatici con lo 0,79\%(2), i contenuti di stampo complottista con lo 
0,40\%(1) e i gameplay con lo 0,40\%(1).

Infine esaminiamo le emozioni provate dal coder durante la visione di ogni video e catalogate in undici categorie: al primo posto 
troviamo l'indignazione con il 27,78\% (70), la felicità/risate con il 23,81\% (60), l'ammirazione con il 17,86\% (45), i contenuti 
sconosciuti e non analizzabili con il 8,73\% (22), la paura/rabbia/tristezza con il 7,54\% (19), l'hype con il 3,97\% (10), 
il sospetto con il 3,97\% (10), lo stupore con il 2,38\% (6), la nostalgia con l'1,59\% (4), l'informazione con l'1,59\% (4) e 
la speranza con il 0,79\% (2).

Una volta analizzati i risultati del processo di categorizzazione dei diversi video ci concentriamo sull'analisi degli hashtag riportati
all'interno delle caption che sono stati raccolti e catalogati separatamente. In totale sono stati raccolti 907 hashtag, di cui 630 
sono unici; i dieci hashtag più popolari sono: \#perte (19), \#fyp (17), \#viral (16), \#putin (15), \#russia (15), \#foryou (13), 
\#ucraina (13), \#guerra (12), \#nato (10), \#neiperte (10). 
Come possiamo ben vedere gli hashtag più popolari riguardano la pagina "per te" di TikTok, questo è dovuto alla tendenza degli utenti 
di voler massimizzare le statistiche dei loro video "sfruttando" l'algoritmo della piattaforma a loro favore, fra i diversi hack 
contro l'algoritmo uno molto popolare è proprio l'utilizzo strategico degli hashtag.
Questo concetto può essere esteso anche ai restanti hashtag che trattano il tema della guerra in Ucraina, un tema decisamente potente 
ed efficace per sfruttare l'algoritmo ed avere una maggiore visibilità.

Guardando le statistiche raccolte emerge che i video hanno un numero maggiore di like, commenti e condivisioni quando al loro interno 
sono presenti sottotitoli rispetto a quando non sono utilizzati; analogamente i video contenenti una qualsiasi lingua parlata 
hanno un maggior numero di commenti e condivisioni, la stessa cosa si nota nei video in cui è presente una musica di sottofondo i 
quali hanno molti più commenti e like rispetto ai video in cui non è utilizzata alcuna musica. 
Al contrario la presenza di testo all'interno del video non influisce in alcun modo sulle metriche riscontrando un numero 
di like, commenti e condivisioni maggiori quando il testo non è presente; questo fenomeno si scontra con quanto visto 
in precedenza considerando che più della metà di tutti i video contiene al suo intero del testo.
Infine è interessante notare come i video contenenti emozioni allarmanti e negative abbiano ricevuto complessivamente 
un numero maggiore di like e commenti rispetto a video con emozioni positive.

Successivamente prendiamo in esame i valori dello user engagement in correlazione con il tempo passato dalla pubblicazione 
dei video, da 83 fino a 90 giorni. 
Come si può immaginare il numero di views e di like tende ad aumentare maggiore è la distanza temporale di pubblicazione (correlazione 
positiva dello 0.25 e 0.14); al contrario il numero di condivisioni rimane quasi del tutto costante (correlazione dello 0.02) e 
addirittura il numero di commenti tende a diminuire (correlazione negativa dello -0.3).

Infine approfondiamo i video in relazione alle loro condivisioni su Twitter. 
Nel periodo di tempo preso in considerazione i 301 video sono stati condivisi su Twitter da 146 account unici, 
fra questi tre account che hanno condiviso un numero maggiore di video, rispettivamente 20, 16 e 11 video (Media = 2.04, SD = 2.51). 
Di seguito concentriamoci solo sui dieci video più condivisi su Twitter; in questi il formato video più condiviso è quello 
dell'infografica (30\%), il contenuto video è quello della guerra in Ucraina (40\%) mentre l'emozione più condivisa è 
l'indignazione (40\%). 
Il numero di like, commenti, condivisioni e views è positivamente correlata con il numero di condivisioni su Twitter 
(0,38, 0,18, 0,57, 0,50), mentre la durata del video non sembra avere alcuna relazione. 
Fra i dieci video solo uno contiene sottotitoli, mentre quasi tutti contengono del testo, una lingua parlata e la caption, 
la musica è presente in meno della metà dei video. 

\section{Discussione}

Questo studio ha esplorato una serie di video di TikTok condivisi sul social network Twitter nel periodo di tempo di sette giorni 
verso l'inizio del mese di marzo 2022. 
Ogni video è stato classificato riguardo il formato, il contenuto e le emozioni che trasmette, inoltre 
sono stati raccolti e correlati alcuni attributi indicativi dello user engagement tra i quali: numero di views, 
numero di like, di commenti e di condivisioni.

\subsection{Formato video e user engagement}

La maggior parte dei video di TikTok fa largo utilizzo degli hashtag, dalle nostre analisi un numero maggiore di hashtag è 
correlato ad un crescente numero di like.
Recenti ricerche sui Social Media hanno mostrato come l'utilizzo degli hashtag permetta agli utenti di trovare più facilmente i 
contenuti\cite{chang2010new}; una ricerca analoga svolta su Twitter ha mostrato come un utilizzo maggiore degli hashtag porti 
ad un numero più alto di retweet\cite{saxton2015advocatingforchange}.
L'analisi dei differenti hashtag inseriti nelle caption nei video presi in esame mostra che i più popolari hanno una relazione stretta 
con la piattaforma, in particolare con la pagina "per te" di TikTok, alcuni esempi sono \#fyp, \#viral e \#perte come a voler aumentare 
le chance del video di essere inserito nella sezione "per te" degli utenti. 
I restanti hashtag molto popolari riguardano invece il tema della guerra in Ucraina, un tema caldo sopratutto in quei giorni presi in 
esame poiché l'invasione russa era iniziata da poche settimane.

Come visto nella Sezione \ref{risultati} la presenza di testo all'interno dei video è molto alta come lo è quella delle caption, infatti 
quasi ogni video ne ha una, molto popolare è anche l'utilizzo di lingue parlate e di musica di sottofondo, mentre i sottotitoli non 
sono così tanto presenti. Le statistiche al contrario hanno mostrato che quei pochi video che contengono sottotitoli hanno un numero 
maggiore di like, commenti e condivisioni, la presenza di una lingua parlata e di una musica di sottofondo si rivelano altrettanto 
importanti per i valori statistici, contrapposta alla presenza di testo che non influisce minimamente.

La tipologia di video più ricorrente è quella del parlato, seguita dalle compilation, le slideshow e le infografiche; i balletti di TikTok e 
l'acting per quanto due dei formati più gettonati nella piattaforma non sono presenti in numero tanto grande e sono stati accorpati in 
un'unica categoria. Probabilmente non si prestano molto bene ad essere condivisi all'esterno della piattaforma dove non è presente lo stesso 
contesto sociale.

La correlazione dei dati statistici con il tempo passato dalla pubblicazione dei video mostra una correlazione positiva rispetto al numero 
di visualizzazioni e di like, questo fenomeno è del tutto atteso poiché video pubblicati prima hanno avuto più tempo per raggiungere 
un numero maggiore di persone; il tempo non è l'unico fattore che incide sulla fama di un video per questo motivo si possono avere 
video più recenti in grado di performare meglio, questo si può notare nelle correlazioni nulle e negative del numero di commenti e 
condivisioni.

\subsection{Contenuti video ed emozioni}

Il contenuto video più popolare è decisamente quello della guerra in Ucraina, come detto precedentemente è un tema scottante sopratutto nel 
periodo in cui sono stati raccolti i dati ed è normale avere una grande quantità di video che ne parlano; in maniera del tutto 
inaspettata troviamo al secondo posto in gradimento i video a tema gossip e personaggi famosi, la maggior parte di questi verte con gruppi 
musicali sudcoreani, come ad esempio i BTS, oppure sono spezzoni di litigi e momenti imbarazzanti vissuti 
dai partecipanti del Grande Fratello VIP, in onda in quel momento. Al terzo posto troviamo video che parlano di politica interna ed 
estera seguiti a breve distanza da video che parlano di Covid-19 mostrando principalmente un punto di vista No-Vax con 
lamentele all'obbligo vaccinale e all'obbligo di portare le mascherine in vigore in Italia.

L'emozione più popolare che il video suscita nell'osservatore è l'indignazione, questo deriva dal fatto che la maggior parte dei video 
trattano temi relativi alla guerra non con lo scopo di fare informazione, ma al contrario, mostrando un punto di vista principalmente 
filo-Putiniano alimentando il sospetto e la disinformazione.
La seconda emozione più riscontrata è la felicità/risate, alcuni video mostrano direttamente spezzoni comici oppure appositamente 
realizzati per suscitare l'ilarità, 
altri creano felicità motivando l'ascoltatore, altri ancora suscitano uno stato d'animo di wholesomeness tramite video contenenti 
animali carini, eventi passati particolarmente piacevoli oppure eventi caritatevoli.

\subsection{Limitazioni e ricerche future}

Nell'approccio utilizzato in questa indagine è possibile identificare diverse limitazioni; per prima cosa la provenienza dei dati, 
i video analizzati sono ricavati da tweet condivisi su Twitter in un periodo di tempo di sette giorni, questo periodo può essere poco 
esaustivo per estrarre al meglio le tipologie, i contenuti e le tematiche che generalmente hanno successo in un contesto multi piattaforma. 
Il periodo in cui si è svolta la ricerca inoltre è in grado d'introdurre una sorta di bias spostando l'attenzione solo su contenuti a 
tema guerra e, di conseguenza, emozioni negative; nelle ricerche future si dovrebbero raccogliere i dati in un periodo di tempo maggiore e 
prestare attenzione agli eventi di scala globale che avvengono in quel momento. 
La seconda limitazione rilevata riguarda il processo di classificazione dei video che è stato eseguito da un solo coder in diverse sessioni 
nell'arco di alcuni giorni; in futuro sarebbe opportuno ampliare questo processo facendo analizzare gli stessi video a due o più coder 
in modo da discutere e risolvere tutte le discrepanze che possono emergere ed ottenere una classificazione più generale.

Il modo in cui si è calcolata la correlazione dei parametri di user engagement con il numero di giorni trascorsi dalla creazione del 
contenuto risulta essere un'ulteriore restrizione; analisi future potrebbero concentrarsi sulla variazione delle metriche nel tempo eseguendo 
il processo di raccolta ogni giorno in modo da poter vedere come il passare del tempo influisce su di esse.

Un ultimo spunto di ricerca consiste nel capire se i video condivisi su Twitter hanno l'effetto di aumentare lo user engagement nella 
piattaforma originale, ossia TikTok; per questa ricerca si consiglia di impostare uno studio non solo dei contenuti condivisi, ma anche 
di video della stessa tipologia che non sono stati mai condivisi su Twitter in modo da poterli paragonare.
\section{Conclusione}

In questa ricerca si è proposta un'analisi dei contenuti di 301 video pubblicati su TikTok e condivisi su Twitter in un 
periodo di tempo dal 3 Marzo 2022 al 10 Marzo 2022 che ha mostrato le differenti tipologie di video e i contenuti che tendono 
ad essere condivise su piattaforme esterne a quella di caricamento.
Per la raccolta dei dati analitici si è sviluppato un tool in grado di semi-automatizzare il processo, che è stato poi affiancato 
da un operatore aiutandolo a svolgere il laborioso processo di raccolta delle variabili e di classificazione manuale dei contenuti 
e delle emozioni di ogni video.

Lo studio ha rilevato che i video contenenti sottotitoli, testo, lingue parlate e musica di sottofondo hanno molto successo nel 
pubblico ricevendo un grande numero di like, commenti e condivisioni. 
Gli hashtag svolgono un ruolo molto importante nell'aumentare le chance di un video di diventare virale per questo motivo se ne fa 
largo uso.
I formati video più popolari sono quelli contenenti parlato, le compilation di spezzoni presi da altri video e le infografiche. 
I contenuti più gettonati riguardano la guerra in Ucraina, il gossip e la politica, mentre le emozioni che vanno per la maggiore 
sono l'indignazione e la felicità/risate.

\phantomsection
\bibliographystyle{unsrt}
\bibliography{relazione}

\end{document}
