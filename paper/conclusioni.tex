\section{Conclusione}

In questa ricerca si è proposta un'analisi dei contenuti di 301 video pubblicati su TikTok e condivisi su Twitter in un 
periodo di tempo dal 3 Marzo 2022 al 10 Marzo 2022 che ha mostrato le differenti tipologie di video e i contenuti che tendono 
ad essere condivise su piattaforme esterne a quella di caricamento.
Per la raccolta dei dati analitici si è sviluppato un tool in grado di semi-automatizzare il processo, che è stato poi affiancato 
da un operatore aiutandolo a svolgere il laborioso processo di raccolta delle variabili e di classificazione manuale dei contenuti 
e delle emozioni di ogni video.

Lo studio ha rilevato che i video contenenti sottotitoli, testo, lingue parlate e musica di sottofondo hanno molto successo nel 
pubblico ricevendo un grande numero di like, commenti e condivisioni. 
Gli hashtag svolgono un ruolo molto importante nell'aumentare le chance di un video di diventare virale per questo motivo se ne fa 
largo uso.
I formati video più popolari sono quelli contenenti parlato, le compilation di spezzoni presi da altri video e le infografiche. 
I contenuti più gettonati riguardano la guerra in Ucraina, il gossip e la politica, mentre le emozioni che vanno per la maggiore 
sono l'indignazione e la felicità/risate.